\section{Relative Entropy}
Relative Entropy or Kullback–Leibler divergence is a measure of the "amount" of difference that two probability distributions have with one another. It is heuristically defined as:
\begin{equation}
D(p(x) \| q(x)) \equiv \sum_{x} p(x) \log \frac{p(x)}{q(x)} \equiv-H(p(x))-\sum_{x} p(x) \log q(x).
\end{equation}
The characterization as heuristic is based on the fact that we use the conventions:$- p(x) \log 0 = \infty$ for $p(x)>0$ and $0 \log 0 \equiv 0$ in order to do calculations. This entropy was introduced in its general case(using probability measures and writing the continuous form of the formula) by Kullback and Leiber in 1951 \cite{kullback1951information}.
\par
As in the case of Shannon entropy, a similar generalization process by Umegaki allows us to define the quantum version, i.e. generalizing Relative Entropy for operator algebras( more specifically for von Neumann algebras) in \cite{umegaki1962conditional}.
\par 
According to Wilde (\citep{wilde2013quantum}):
\begin{definition}(Quantum Relative Entropy) The quantum relative entropy $D(\rho \| \sigma)$ between density operators $\rho \in \mathcal{D}(\mathcal{H})$ and $\sigma \in \mathcal{L}(\mathcal{H})$ is defined by:
\begin{equation}
D(\rho \| \sigma)= 
 \begin{cases} 
      \operatorname{Tr}[ \rho\log \rho ]-\operatorname{Tr} [ \rho \log \sigma]   & \operatorname{supp}(\rho) \subseteq \operatorname{supp}(\sigma) \\
      \infty & otherwise 
\end{cases}
\end{equation}
\end{definition}
\noindent
These types of functions(that may accept infinity as a value) are characterized as generalized. They can be define with rigor with the use of measure theory(in the quantum case using operator algebras).
\par
To ease our calculations, we can use the same techniques as in \secref{vonNeumannEntropysec} and define:
\begin{definition}(Heuristic Quantum Relative Entropy)The heuristic quantum relative entropy $Q(\rho \| \sigma)$ between the density operators $\rho$ and $\sigma$ is defined by:
\label{heuristickulback}
\begin{equation}
Q(\rho \| \sigma)= S(\rho)-\lim_{\epsilon \to -\infty} \big( \operatorname{Tr}[\rho  G(\epsilon; \sigma)] \big)
\end{equation}
in which $G: [0,1] \rightarrow \mathbb{R}$ as:
\begin{equation}
G(\epsilon;x)=
\begin{cases}
   \ln x   &   x \in (0,1] \\
   \epsilon  &   x=0
\end{cases}
\end{equation}
$\epsilon$ is a free parameter with $\epsilon \in \bbR$.
\end{definition}
\noindent
This redefinition serves our calculation and programming goals sufficiently, at least for symbolic programming.
\begin{note}
The function $G(\epsilon;x)$ satisfies the conditions needed to accept spectral matrix generalization described in \autoref{chap:1}.
\end{note}
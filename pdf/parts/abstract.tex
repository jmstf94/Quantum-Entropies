\newpage
\thispagestyle{plain}

\vspace*{\fill}
\begin{center}
    \vspace{0.9cm}
    \textbf{Abstract}
\end{center}
The current master thesis is a study of quantum entropies in finite quantum systems. Specifically, we use examples of experimentally produced quantum states, to emphasize properties of the various quantum entropies while demonstrating analytic calculations and programming techniques. We start by establishing some important aspects of the spectral theorem. We discuss the modern literature regarding different definitions of entropies, focusing on von Neumann, Renyi and Tsallis entropies, quantum conditional entropies and the quantum relative entropy. Furthermore, we rewrite heuristic forms of the known definitions that have a clear algorithmic interpretation, thus simplifying the programmability of each entropy. Finally, we exploit the described techniques and emphasize properties of quantum entropies in classes of quantum systems such as pure states, mixed states, Werner states and entangled states. 
\vspace*{\fill}

\newpage
\vspace*{\fill}
\begin{center}
    \vspace{0.9cm}
\begin{otherlanguage}{greek}
\textbf{Περίληψη}
\end{otherlanguage}
\end{center}
\begin{otherlanguage}{greek}
Η παρούσα διπλωματική εργασία αφορά την μελέτη κβαντικών εντροπιών σε πεπερασμένα κβαντικά συστήματα. Ειδικότερα, χρησιμοποιούμε παραδείγματα κβαντικών συστημάτων που έχουν πραγματοποιηθεί πειραματικά, τονίζοντας ιδιότητες των διάφορων κβαντικών εντροπιών παρουσιάζοντας αναλυτικούς υπολογισμούς και προγραμματιστικές τεχνικές. Αρχικά, αναφέρουμε κάποιες απαραίτητες πτυχές του φασματικού θεωρήματος. Συζητούμε την σύγχρονη βιβλιογραφία σχετικά με τους διαφορετικούς ορισμούς κβαντικών εντροπιών επικεντρώνοντας στις εντροπίες \textlatin{Renyi, von Neumann, Tsallis}, τις δεσμευμένες κβαντικές εντροπίες και την σχετική κβαντική εντροπία. Στην συνέχεια, γραφούμε ευρετικούς ορισμούς των γνωστών ορισμών ώστε να επιδέχονται αλγοριθμικής ερμηνείας απλοποιώντας έτσι την προγραμματισιμότητα της κάθε εντροπίας. Τελικά, επιστρατεύοντας τις παραπάνω τεχνικές, τονίζουμε ιδιότητες των κβαντικών εντροπιών σε κλάσεις κβαντικών συστημάτων όπως καθαρές καταστάσεις, μεικτές καταστάσεις, καταστάσεις \textlatin{Werner} και διεμπλεγμένες καταστάσεις.
\end{otherlanguage}
\vspace*{\fill}
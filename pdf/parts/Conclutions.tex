\chapter{Conclusions and further work}
\begin{itemize}
\item We have established that there can be experiments that measure entropies or entanglement measures directly. This means that without explicitly preparing the density matrix, we can deduce certain statistical properties of the system that may include entanglement. It is tempting to conjecture that if there can be designed an experiment that measures \ref{qentropy} directly, we can detect and quantify the entanglement of the pure state $\sigma(\theta)$. This claim is based on the plot \ref{figr7} combined with intuition from the conditional quantum entropy calculation of $\sigma(\theta)$ and the theory of entanglement entropy.
\item In \citep{wilde2013quantum} is proven for the quantum relative entropy that:
\begin{equation}
D(\rho \| \sigma)=\lim _{\varepsilon \to 0^{+}} D(\rho \| \sigma+\varepsilon I)
\end{equation}
Thus using a function $G(\varepsilon ;x)=x+\varepsilon$ different than ours for the matrix generalization, and taking the limit to zero. Since different functions $G(\varepsilon;x)$ can be used to heuristically determine the quantum relative entropy some questions rise. What is the class of functions that can be used in these calculations? Can we generally find the conditions needed? Are there any advantages and disadvantages among them? In addition, other functions withing relative entropy definitions like $g-relative$(\citep{holevo2012quantum}) or $f-relative$(\citep{hayashi2016quantum}) require further investigation regarding their connection with the class of functions $G(\epsilon;x)$.
\item Figures \ref{figuridion3d},\ref{figuridiont2},\ref{figuraki6},\ref{figr1},\ref{figr3},\ref{figr4},\ref{figr5},\ref{figr6},\ref{figr7} could not be found in the literature. I stand in the entanglement limit value plots in the Werner conditional entropy calculations, which demonstrate that for larger $q$'s and $a$'s we have smaller values of the conditional entropy. We are not aware of a general proof of this property. Thus raises the question for the generality of this result.
\end{itemize}
%\section{Further work}
%\begin{itemize}
%\item analytical proof of G and other functions
%\end{itemize}
\section{Tsallis entropy}
Tsalis entropy is another extensively used generalization  of common entropies with one parameter (see \cite{tsallis1998},\cite{tsallis2016}). It was originally 
proposed in 1988(\cite{tsallis1988original}) in its classical form as an alternative to \eqref{gibbs}. For a probability mass ${p_i}$:
\begin{equation}
S_{q}\left(p_{i}\right)=\frac{k}{q-1}\left(1-\sum_{i} p_{i}^{q}\right)
\end{equation}
It's quantum generalization was discussed firstly in \cite{tsallisquantum} and was rewritten later as:
\begin{definition}(Quantum Tsallis entropy)For a quantum state $\rho$ the quantum Tsallis entropy is defined as:
\begin{equation}
S_q(\rho)=\frac{1}{1-q}\left(\Tr \left(\rho^{q}\right)-1\right), q \in(0,1) \cup(1, \infty)
\end{equation}
\label{TsallisDEF}
\end{definition}
\noindent
We follow the same logic as we did in the Renyi case and redefine a heuristic form of the quantum Tsallis entropy:
\begin{definition}(Heuristic quantum Tsallis entropy)For a density matrix $\rho \in \mathcal{D}(\mathcal{H})$ the heuristic form of the quantum Tsallis entropy is defined as:
\begin{equation}
T(q ; \rho)= \frac{1}{1-q}\left(\Tr \left[t(q;\rho)\right]-1\right), q \in(0,1) \cup(1, \infty)
\end{equation}
in which $t$ is the function $t:[0,1] \rightarrow \mathbb{R^{+}}$:
\begin{equation}
t(q;x)=x^{q}
\end{equation}
\label{tsallis}
\end{definition}
\begin{note}
It can be proved that: $S(\rho)=\lim _{q \rightarrow 1} T(q;\rho)$.
\end{note}